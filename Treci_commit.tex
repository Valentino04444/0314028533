\documentclass{article}
\usepackage[utf8]{graphicx} % Required for inserting images


\title{IT i Primjena - Računalo}
\author{Valentino Andrijević}
\date{Siječanj 2024}

\begin{document}
\maketitle
\tableofcontents
\newpage
\section{Uvod}
Ovo računalo je sagrađeno specificno za igranje igrica i komponente su skupe. Cilj je dobiti računalo koje
će biti dugotrajno i koje se može koristiti za bilo koji posao koji korisnik želi odraditi.

\section{Procesor}
\textbf{AMD Ryzen 7 7800X3D 4.2 GHz 8-core}
\\ Max memorija: 128GB
\\ Integrirana grafike: Radeon
\\ Izvedba L1 Cache
\\ 8 x 32 kB Instruction
\\ 8 x 32 kB Data
\\ Izvedba L2 Cache
\\ 8 x 1 MB
\\ L3 Cache
\\ 1 x 96 MB
\\ \rule{\textwidth}{0.5pt}
\\ Ovaj procesor je među najboljim procesorima od Ryzen-a.
U testovima je pobjedio Intelov Core i9-13900K.

\section{Procesorski hladnjak}
\textbf{Cooler Master MasterAir MA824 Stealth}
\\ Glasnoća: 0-24.6 dB
\\ Visina: 165 mm
\\ Boja: crna
\\ \rule{\textwidth}{0.5pt}
\\ Ovaj procesorski hladnjak je poznat po svojoj efikasnosti u hlađenju gaming procesora.
Cijena mu je vrlo visoka ali vrlo dobro obavlja svoj posao hlađenjem vrlo jakog procesora.

\section{Matična ploča}
\textbf{ASUS ROG Strix X670E-E}
\\ Gaming Socket AM5(LGA 1718) Ryzen 7000 ATX Motherboard(18+2 Power Stages,PCIe® 5.0, DDR,4xM.2 Slots,USB 3.2 Gen 2x2, WiFi 6E,PCIe Slot Q-Release, M.2 Q-Latch).
\\ \rule{\textwidth}{0.5pt}
\\ Ova matična ploča je moderna i nova matična koja podržava procesor kojeg sam uzeo. Matična je poznata kao matična za novije Ryzen procesore. Kako bi ova matična podržala procesor koji sam uzeo mora joj BIOS biti ažuriran na verziju 1303 ili više.

\section{Radna Memorija}
\textbf{ADATA XPG Lancer Blade}
\\ 32 GB (2 x 16 GB) DDR5-6000 CL30 Memory
\\ \rule{\textwidth}{0.5pt}
\\ Ova memorija je vrlo brza memorija koja podržava 6000MT/s kroz EXPO. Kompatibilna je sa procesorom kojeg sam uzeo. Idealna je za igranje igrica i brze izvedbe.

\section{Trajna memorija}
\subsection{SSD}
\\ \textbf{1. WD Black SN850X}
\\ Kapacitet: 1TB
\\ Form Factor: M.2-2280
\\ Interface: M.2 PCIe 4.0 X4
\\ \rule{\textwidth}{0.5pt}
\\ Ovaj SSD je vrlo brz i efikasan što je poželjno za video igrice i za bilo koje procese koji se trebaju brzo obaviti na računalu.

\subsection{HDD}
\\ \textbf{Seagate Barracuda (3TB)}
\\ Kapacitet: 3TB
\\ Tip: 7200 RPM
\\ Cache: 64 MB
\\ SATA 6.0 Gb/s
\\ \rule{\textwidth}{0.5pt}
\\ Ovaj HDD je vrlo brz i pouzdan. Ima dovoljno prostora za igrice i medije.

\section{Grafička kartica}
\textbf{NVIDIA Founders Edition GeForce RTX 4090 24 GB Video Card}
\\ Memorija: 24 GB
\\ Tip memorije: GGDR6x
\\ Core Clock: 2230 MHz
\\ Boost Clock: 2520 MHz
\\ Boja: Black/Silver
\\ Dužina 304 mm
\\ TDP: 450 W
\\ Hlađenje: 2 ventilatora
\\ PCIe x16
\\ \rule{\textwidth}{0.5pt}
\\ Ova grafička kartica je trenutno među najboljim i najbržim grafičkim karticama od NVIDIA. Izuzetno je brza, efikasna, glasnoća joj je umjerena a izvedba na video igricama je izuzetno dobra. Uz ovu grafičku karticu nitko neće imati problema sa izvedbom video igrica.

\section{Kućište}
\textbf{Corsair 4000D Airflow ATX Mid Tower Case}
\\ Front Panel USB
\\ USB 3.2 Gen 2 Type-C
\\ USB 3.2 Gen 1 Type-A
\\ Podržava matičnu koju sam odabrao
\\ Maksimalna duljina video kartice: 360 mm
\\ Dimenzije: 453 mm x 230 mm x 466 mm
\\ Volumen: 48.553 L
\\ \rule{\textwidth}{0.5pt}
\\ Kućište omogućuje dovoljno prostora za postavljanje svih komponenti računala i omogućuje dobar protok zraka

\section{Napajanje}
\textbf{Corsair RM850x (2021)}
\\ 850W
\\ Dužina: 160 mm
\\ EPS/ATX Connectors: 3 x EPS 8-pin
\\ PCIe 6+2-Pin Connectors: 4
\\ SATA Connectors: 14
\\ Molex 4-Pin Connectors: 4
\\ Tip: ATX
\\ Ocjena efikasnosti: 80+ Gold
\\ \rule{\textwidth}{0.5pt}
\\ Ovo napajanje je izuzetno efikasno i ima vrlo dobre recenzije. Podržava ostale komponente u računalu i pruža vrlo jako napajanje koje je potrebno za ostale komponente za maksimalnu brzinu.


\section{Operativni sustav}
\textbf{Microsoft Windows 11 Pro OEM - DVD 64-bit}
\\ \rule{\textwidth}{0.5pt}
\\ Ova verzija Windowsa je dovoljna za igrice i opću upotrebu računala.

\section{Monitor}
\textbf{Acer XV252QF}
\\ Veličina 24.5 inča
\\ Rezolucija: 1920x1080
\\ Brzina osvježavanja: 390 Hz
\\ Vrijeme odaziva: 1 ms
\\ Tip panela: IPS
\\ Frame Sync: FreeSync Premium
\\ Kut gledanja: 178° H x 178° V
\\ Ulaz: 1 x HDMI, 1 x DisplayPort
\\ \rule{\textwidth}{0.5pt}
\\ Ovaj monitor je kvalitetan, brz i vizualno privlačan. Savršen je za igranje video igrica uz visoku brzinu osvježavanja od čak 390 Hz.

\section{Miš}
\textbf{Logitech G Pro X Superlight}
\\ Težina: 63 g
\\ Senzor: HERO 25K
\\ Glideri: PTFE
\\ \rule{\textwidth}{0.5pt}
\\ Ovaj miš je jedan od najboljih miševa za gaming. Jako je pouzdan, lagan i ima izvrstan senzor

\section{Tipkovnica}
\textbf{MOD007B PC Santorini}
\\ Bluetooth 5.0
\\ 2.4 Ghz
\\ Type C
\\ Actuation distance: 0.2 mm
\\ \rule{\textwidth}{0.5pt}
\\ Ova tipkovnica je osobna preferenca, tipkovnica je izvrsna za tipkanje i za igranje igrica. Proizvodi jako ugodan zvuk tipkanja i sila potrebna za tipkanje je vrlo mala.
\\
\\ \rule{\textwidth}{0.5pt}
\\ \rule{\textwidth}{0.5pt}
\\
\\ \textbf{Cijene}
\\

\begin{tabular}{c|c|c}
    Procesor & AMD Ryzen 7 7800X3D 4.2 GHz 8-Core Processor & 389\$\\
    Procesorski hladnjak & Cooler Master MasterAir MA824 Stealth & 99\$\\
    Matična ploča & Asus ROG STRIX X670E-E & 459\$\\
    Radna memorija & ADATA XPG Lancer Blade 32 GB & 184\$\\
    HDD & Seagate BarraCuda 3 TB & 94.49\$\\
    SSD & Western Digital Black SN850X & 84.99\$\\
    Grafička kartica & NVIDIA Founders Edition GeForce RTX 4090 & 2231\$\\
    Kućište & Corsair 4000D Airflow ATX Mid Tower Case & 105\$\\
    Napajanje & Corsair RM850x & 130\$\\
    Operacijski sustav & Microsoft Windows 11 Pro OEM & 148.99\$\\
    Monitor & Acer Nitro XV252QF 24.5" & 613\$\\
    Miš & Logitech G Pro X Superlight & 130\$\\
    Tipkovnica & MOD007B PC Santorini & 150\$\\
    
\end{tabular}

\end{document}
