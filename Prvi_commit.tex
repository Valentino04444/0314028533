\documentclass{article}
\usepackage[utf8]{graphicx} % Required for inserting images


\title{IT i Primjena - Računalo}
\author{Valentino Andrijević}
\date{Siječanj 2024}

\begin{document}

\maketitle
\section{Uvod}
Ovo računalo je sagrađeno specificno za igranje igrica i komponente su skupe. Cilj je dobiti računalo koje
će biti dugotrajno i koje se može koristiti za bilo koji posao koji korisnik želi odraditi.

\section{Procesor}
\textbf{AMD Ryzen 7 7800X3D 4.2 GHz 8-core}
\\ Max memorija: 128GB
\\ Integrirana grafike: Radeon
\\ Izvedba L1 Cache
\\ 8 x 32 kB Instruction
\\ 8 x 32 kB Data
\\ Izvedba L2 Cache
\\ 8 x 1 MB
\\ L3 Cache
\\ 1 x 96 MB
\\ Ovaj procesor je među najboljim procesorima od Ryzen-a.
U testovima je pobjedio Intelov Core i9-13900K.

\section{Procesorski hladnjak}
\textbf{Cooler Master MasterAir MA824 Stealth}
\\ Glasnoća: 0-24.6 dB
\\ Visina: 165 mm
\\ Boja: crna
\\ Ovaj procesorski hladnjak je poznat po svojoj efikasnosti u hlađenju gaming procesora.
Cijena mu je vrlo visoka ali vrlo dobro obavlja svoj posao hlađenjem vrlo jakog procesora.

\section{Matična ploča}
\textbf{ASUS ROG Strix X670E-E}
\\ Gaming Socket AM5(LGA 1718) Ryzen 7000 ATX Motherboard(18+2 Power Stages,PCIe® 5.0, DDR,4xM.2 Slots,USB 3.2 Gen 2x2, WiFi 6E,PCIe Slot Q-Release, M.2 Q-Latch).
\\ Ova matična ploča je moderna i nova matična koja podržava procesor kojeg sam uzeo. Matična je poznata kao matična za novije Ryzen procesore.

\section{Memorija}
\textbf{ADATA XPG Lancer Blade}
\\ 32 GB (2 x 16 GB) DDR5-6000 CL30 Memory
\\ Ova memorija je vrlo brza memorija koja podržava 6000MT/s kroz EXPO. Kompatibilna je sa procesorom kojeg sam uzeo. Idealna je za igranje igrica i brze izvedbe.

\section{}

\end{document}
